\documentclass{article}
\usepackage{graphicx} % Required for inserting images

\title{Lab3 Answers}
\author{Youssef A}
\date{September 2023}

\begin{document}

\maketitle

\section{Explain the difference between internal and external fragmentation.}
The difference between internal and external fragmentation is that internal fragmentation only happens when a block of memory is allocated, but not all of it is used by a process.
External fragmentation happens when free memory is broken up into small blocks of data in a memory,  and because of this using up memory can be a problem as your system can't detect free memory even if there is. 

\section{Given five (5) memory partitions of 100KB, 500KB, 200KB, 300KB, and 600KB (in that order), how would optimal, first-fit, best-fit, and worst-fit algorithms place processes of 212KB, 417KB, 112KB, and 426KB (in that order)?}

\textbf{First-Fit:} First-fit places each process in the first available memory partition so 212K is put in 500K partition, then 417K is put in 600K partition. 112K is put in the 288K partition (remaining from the 500K after 212K), and finally 426K has to wait or be rejected.

\textbf{Best-Fit:} Best-fit places each process in the smallest available partition, so 212KB is put into 300K partition, 417KB is put into 500KB partition. 112K is put into 200KB partition, and finally 426KB is put into 600KB partition. 

\textbf{Worst-Fit:} Worst-fit places each process in the largest partition that's available, so 212KB is placed into 600KB partition. 417KB is placed into 500KB partition. 112KB is placed into the 388KB partition (remaining from the 600K after 212K), and lastly 426K has to wait or be rejected. 

% Comment: used this to help: https://faculty.ksu.edu.sa/sites/default/files/tutorial_8-sol.pdf

\end{document}
